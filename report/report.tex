%----------------------------------------------------------------------------------------
%	PACKAGES AND OTHER DOCUMENT CONFIGURATIONS
%----------------------------------------------------------------------------------------

\documentclass{article}

\usepackage{fancyhdr} % Required for custom headers
\usepackage{lastpage} % Required to determine the last page for the footer
\usepackage{extramarks} % Required for headers and footers
\usepackage{graphicx} % Required to insert images
\usepackage{lipsum} % Used for inserting dummy 'Lorem ipsum' text into the template
\usepackage{enumitem}
\usepackage{multicol}

\usepackage{amsmath}


% Margins
\topmargin=-0.45in
\evensidemargin=0in
\oddsidemargin=0in
\textwidth=6.5in
\textheight=9.0in
\headsep=0.25in 

\linespread{1.1} % Line spacing

% Set up the header and footer
\pagestyle{fancy}
\lhead{\hmwkAuthorName} % Top left header
\chead{\hmwkClass\ \hmwkTitle} % Top center header
\rhead{\firstxmark} % Top right header
\lfoot{\lastxmark} % Bottom left footer
\cfoot{} % Bottom center footer
\rfoot{Page\ \thepage\ of\ \pageref{LastPage}} % Bottom right footer
\renewcommand\headrulewidth{0.4pt} % Size of the header rule
\renewcommand\footrulewidth{0.4pt} % Size of the footer rule

\setlength\parindent{0pt} % Removes all indentation from paragraphs

%----------------------------------------------------------------------------------------
%	DOCUMENT STRUCTURE COMMANDS
%	Skip this unless you know what you're doing
%----------------------------------------------------------------------------------------

% Header and footer for when a page split occurs within a problem environment
\newcommand{\enterProblemHeader}[1]{
\nobreak\extramarks{#1}{#1 continued on next page\ldots}\nobreak
\nobreak\extramarks{#1 (continued)}{#1 continued on next page\ldots}\nobreak
}

% Header and footer for when a page split occurs between problem environments
\newcommand{\exitProblemHeader}[1]{
\nobreak\extramarks{#1 (continued)}{#1 continued on next page\ldots}\nobreak
\nobreak\extramarks{#1}{}\nobreak
}

\setcounter{secnumdepth}{0} % Removes default section numbers
\newcounter{homeworkProblemCounter} % Creates a counter to keep track of the number of problems

\newcommand{\homeworkProblemName}{}
\newenvironment{homeworkProblem}[1][Problem \arabic{homeworkProblemCounter}]{ % Makes a new environment called homeworkProblem which takes 1 argument (custom name) but the default is "Problem #"
\stepcounter{homeworkProblemCounter} % Increase counter for number of problems
\renewcommand{\homeworkProblemName}{#1} % Assign \homeworkProblemName the name of the problem
\section{\homeworkProblemName} % Make a section in the document with the custom problem count
\enterProblemHeader{\homeworkProblemName} % Header and footer within the environment
}{
\exitProblemHeader{\homeworkProblemName} % Header and footer after the environment
}

\newcommand{\problemAnswer}[1]{ % Defines the problem answer command with the content as the only argument
\noindent\framebox[\columnwidth][c]{\begin{minipage}{0.98\columnwidth}#1\end{minipage}} % Makes the box around the problem answer and puts the content inside
}

\newcommand{\homeworkSectionName}{}
\newenvironment{homeworkSection}[1]{ % New environment for sections within homework problems, takes 1 argument - the name of the section
\renewcommand{\homeworkSectionName}{#1} % Assign \homeworkSectionName to the name of the section from the environment argument
\subsection{\homeworkSectionName} % Make a subsection with the custom name of the subsection
\enterProblemHeader{\homeworkProblemName\ [\homeworkSectionName]} % Header and footer within the environment
}{
\enterProblemHeader{\homeworkProblemName} % Header and footer after the environment
}
   
%----------------------------------------------------------------------------------------
%	NAME AND CLASS SECTION
%----------------------------------------------------------------------------------------

\newcommand{\hmwkTitle}{Assignment\ \#2} % Assignment title
\newcommand{\hmwkDueDate}{Wednesday,\ October 19,\ 2016} % Due date
\newcommand{\hmwkClass}{15-887} % Course/class
\newcommand{\hmwkClassTime}{1:30pm} % Class/lecture time
\newcommand{\hmwkClassInstructor}{Likhaechev} % Teacher/lecturer
\newcommand{\hmwkAuthorName}{Jordan Ford} % Your name

%----------------------------------------------------------------------------------------
%	TITLE PAGE
%----------------------------------------------------------------------------------------

\title{
\vspace{2in}
\textmd{\textbf{\hmwkClass:\ \hmwkTitle}}\\
\normalsize\vspace{0.1in}\small{Due\ on\ \hmwkDueDate}\\
\vspace{3in}
}

\author{\textbf{\hmwkAuthorName}}
\date{} % Insert date here if you want it to appear below your name

%----------------------------------------------------------------------------------------

\begin{document}

\maketitle

%----------------------------------------------------------------------------------------
%	TABLE OF CONTENTS
%----------------------------------------------------------------------------------------

%\setcounter{tocdepth}{1} % Uncomment this line if you don't want subsections listed in the ToC

\newpage
\tableofcontents
\newpage

%----------------------------------------------------------------------------------------
%	PROBLEM 1
%----------------------------------------------------------------------------------------

% To have just one problem per page, simply put a \clearpage after each problem

\begin{homeworkProblem}
%\vspace{10pt} % Question

\begin{homeworkSection}{(a)} % Section within problem

To run this planner, please use the following command:
\textbf{python world.py -a prob1.txt}\\

This planner performs a forward djikstra search in x and y and uses the result as an informed heuristic for a single backward A* search in x, y, and t.\\

The 2D djikstra search expands all nodes in the graph - one million nodes for test case one. On my machine, it takes about 8 seconds to complete this phase of the search. The distances calculated by the djikstra search are then used as the heuristic for a backward A* search which searches in both position and time. \\

\textbf{TODO: Time, cost, number of states expanded}

\end{homeworkSection}

\begin{homeworkSection}{(b)} % Section within problem

To run this planner, please use the following command:
\textbf{python world.py -b prob1.txt}\\

One of the advantages of the algorithm I chose for part (a) is its ease of extension to part (b). By weighting the backward A* search, it is easy to trade optimality for speed, and it comes with the typical weighted A* performance guarantees.\\


\textbf{TODO: Time, cost, number of states expanded}

\end{homeworkSection}

\end{homeworkProblem}

%----------------------------------------------------------------------------------------
%	PROBLEM 2
%----------------------------------------------------------------------------------------

% To have just one problem per page, simply put a \clearpage after each problem

\begin{homeworkProblem}
%\vspace{10pt} % Question

\begin{homeworkSection}{2.1} % Section within problem
Suppose you have two consistent heuristic functions: $h_1$ and $h_2$. Prove that h(s) = max($h_1(s)$, $h_2(s)$) for all states s in the graph is also a consistent heuristic function.\\
\problemAnswer{ % Answer
\vspace{1mm}
A heuristic is consistent if
\begin{equation*}
h(n) \leq c(n, n') + h(n')
\end{equation*}
for every node $n$ and its child node $n'$.\\

Proof:
\begin{align*}
h(n) &= max( h_1(n), h_2(n) ) \\
     &\leq max( c(n, n') + h_1(n'), c(n, n') + h_2(n')) \\
     &\leq c(n, n') + max(h_1(n'), h_2(n')) \\
     &\leq c(n, n') + h(n')
\end{align*}
}
\\
Suppose you have two consistent heuristic functions: $h_1$ and $h_2$. Prove that h(s) = min($h_1(s)$, $h_2(s)$) for all states s in the graph is also a consistent heuristic function.\\
\problemAnswer{ % Answer
\vspace{1mm}
A heuristic is consistent if
\begin{equation*}
h(n) \leq c(n, n') + h(n')
\end{equation*}
for every node $n$ and its child node $n'$.\\

Proof:
\begin{align*}
h(n) &= min( h_1(n), h_2(n) ) \\
     &\leq min( c(n, n') + h_1(n'), c(n, n') + h_2(n')) \\
     &\leq c(n, n') + min(h_1(n'), h_2(n')) \\
     &\leq c(n, n') + h(n')
\end{align*}
}
\end{homeworkSection}

\begin{homeworkSection}{2.2}
\textbf{d.} Monotonically non-increasing sequence
\end{homeworkSection}

\begin{homeworkSection}{2.3}
\textbf{f.} None of the above
\end{homeworkSection}

\begin{homeworkSection}{2.4}
\textbf{e.} None of the above
\end{homeworkSection}

\end{homeworkProblem}



%----------------------------------------------------------------------------------------

\end{document}
